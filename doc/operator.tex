\documentclass[12pt]{article}

\usepackage{sectsty}
\usepackage{graphicx}
\usepackage{cite}
%\usepackage{bookman}
%\usepackage{times}
\usepackage{url}
\usepackage{lscape}
\usepackage{indentfirst}
\usepackage{latexsym}
\usepackage{multirow}
\usepackage{tabls}
\usepackage{wrapfig}
\usepackage{longtable}
\usepackage{supertabular}
\usepackage{caption}
\usepackage{subcaption}
\usepackage{amsmath}
\usepackage{amssymb}
\usepackage{setspace}
\usepackage[bookmarks, colorlinks=true, plainpages=false, citecolor=blue, urlcolor=blue, filecolor=blue, linkcolor=blue]{hyperref} 
\usepackage[font={footnotesize,stretch=1.0}]{caption}

\allsectionsfont{\centering}
\newcommand{\pd}[2]{\frac{\partial #1}{\partial #2}}
\newcommand{\gyav}[1]{\left\langle #1 \right\rangle_\mathbf{R}}
\newcommand{\gyavalt}[1]{\left\langle #1 \right\rangle_\mathbf{r}}
\newcommand{\fsav}[1]{\left\langle #1 \right\rangle_\psi}
\newcommand{\vol}{\mathcal{V}}
\newcommand{\turbav}[1]{\left\langle #1 \right\rangle_t}
\newcommand{\muav}[1]{\left\langle #1 \right\rangle_\mu}
\newcommand{\vdrift}{\mathbf{v}_{\mathrm{d}s}}
\newcommand{\en}{E}
\newcommand{\jacob}{\mathcal{J}}
\newcommand{\jacobv}{\mathcal{J}_v}
\newcommand{\order}[1]{\mathcal{O}\left( #1 \right)}
\newcommand{\dv}{\, \mathrm{d}^3 \mathbf{v}}
\newcommand{\dr}{\, \mathrm{d}^3 \mathbf{r}}
\newcommand{\curlof}{\nabla \times}
\newcommand{\divof}{\nabla \cdot}
\newcommand{\psimin}{\psi_\mathrm{min}}
\newcommand{\psimax}{\psi_\mathrm{max}}
\newcommand{\rhat}{\hat{r}}
\newcommand{\vhat}{\hat{v}}
\newcommand{\Dref}{D_\mathrm{ref}}
\newcommand{\vref}{v_\mathrm{ref}}
\newcommand{\kpar}{{k_\|}}
\newcommand{\vperp}{\mathbf{v}_\perp}
\newcommand{\half}{\tfrac{1}{2}}
\newcommand{\vchi}{\mathbf{v}_\chi}
\newcommand{\vphi}{\mathbf{v}_\phi}
\newcommand{\ExB}{\mathbf{E} \times \mathbf{B}}
\newcommand{\kev}{\,\mathrm{keV}}
\newcommand{\mev}{\,\mathrm{MeV}}
\newcommand{\Ealpha}{E_\alpha}
%\newcommand{\eqref}{\left( #1 \right)}




\title{Description of matrix operator for finite-volume form of advection-diffusion equation}

\author{George J. Wilkie}

\begin{document}
\maketitle
%\tableofcontents

%\section{Introduction}

We seek to descritize an operator of the form:
\begin{equation} \label{maineqn}
   M\left[f \right] = \frac{1}{J} \pd{ }{x} \left( J \Gamma \right) =  \frac{1}{J} \pd{ }{x} \left[ J A f + J D  \pd{f}{x} \right]
\end{equation}
using a finite-volume method with piecewise constant elements. Each element has a average/constant value of $f$ and $x$ given by $f_i$ and $x_i$, respectively, for $1 \leq i \leq N$. The fluxes $\Gamma_{i+1/2}$ and $\Gamma_{i-1/2}$ are defined on the mesh faces $x_{i+1/2}$ and $x_{i-1/2}$. The Jacobian of the (arbitrary) coordinate $x$ is $J(x)$. Using the average of the adjacent cells to represent the value of $f$ at the cell boundary, we can descritize Eq. \ref{maineqn} as:
\begin{align}
   M\left[ f_i \right] \approx& \frac{1}{J_i} \frac{1}{x_{i+1/2} - x_{i-1/2}} \left[ \left(JA\right)_{i+1/2} \frac{f_{i+1} + f_i }{2}  -  \left(JA\right)_{i-1/2} \frac{ f_{i} + f_{i-1} }{2}   \right. \nonumber\\
   & \left. + \left(JD \right)_{i+1/2}\frac{ f_{i+1} - f_i }{v_{i+1}-v_i} - \left(JD \right)_{i-1/2} \frac{ f_{i} - f_{i-1} }{v_{i}-v_{i-1}}  \right]  \\
   =&  a_{i-1} f_{i-1} +  b_i f_i + c_{i+1} f_{i+1}  .
\end{align}
The tridiagonal coefficients for interior points are given by:
\begin{equation}
   a_{i-1} = \frac{1}{J_i \left( x_{i+1/2} - x_{i-1/2} \right)} \left[- \frac{1}{2}\left(JA\right)_{i-1/2} + \frac{\left(JD\right)_{i-1/2} }{ x_{i}-x_{i-1}} \right]
\end{equation}
\begin{equation}
   b_{i} = \frac{1}{J_i \left( x_{i+1/2} - x_{i-1/2} \right)} \left[ \frac{1}{2}\left(JA\right)_{i+1/2} - \frac{1}{2}\left(JA\right)_{i-1/2} - \frac{\left(JD\right)_{i+1/2} }{ x_{i+1}-x_i} - \frac{\left(JD\right)_{i-1/2} }{ x_{i}-x_{i-1}} \right]
\end{equation}
\begin{equation}
   c_{i+1} = \frac{1}{J_i \left( x_{i+1/2} - x_{i-1/2} \right)} \left[ \frac{1}{2}\left(JA\right)_{i+1/2} + \frac{\left(JD\right)_{i+1/2} }{ x_{i+1}-x_i} \right]
\end{equation}

For Dirichlet boundary conditions, these are replaced by unity on the diagonal element, and zeros elsewhere. For flux-specified boundary conditions at $i=1$:
\begin{equation}
   M\left[ f_1 \right] \approx \frac{1}{J_1} \frac{1}{x_{3/2} - x_{1/2}} \left[ \left(JA\right)_{3/2} \frac{f_{2} + f_1 }{2}  + \left(JD \right)_{3/2}\frac{ f_{2} - f_1 }{v_{2}-v_1}   \right]  
\end{equation}
and
\begin{equation}
   b_{1} = \frac{1}{J_1 \left( x_{3/2} - x_{1/2} \right)} \left[ \frac{1}{2}\left(JA\right)_{3/2} - \frac{\left(JD\right)_{3/2} }{ x_{2}-x_1} \right]
\end{equation}
\begin{equation}
   c_{2} = \frac{1}{J_1 \left( x_{3/2} - x_{1/2} \right)} \left[ \frac{1}{2}\left(JA\right)_{3/2} + \frac{\left(JD\right)_{3/2} }{ x_{2}-x_1} \right],
\end{equation}
while the incoming flux $\Gamma_{1/2} = \Gamma\left(x_{1/2}\right)$ is given and the following must be added to the source:
\begin{equation}
   \Delta S_1  = \frac{J_{1/2} \Gamma_{1/2}}{J_1 \left( x_{3/2} - x_{1/2} \right)}.
\end{equation}
For flux conditions on the right side, we have similarly:
\begin{equation}
   M\left[ f_N \right] \approx \frac{1}{J_N} \frac{1}{x_{N+1/2} - x_{N-1/2}} \left[ -  \left(JA\right)_{N-1/2} \frac{ f_{N} + f_{N-1} }{2}  - \left(JD \right)_{N-1/2} \frac{ f_{N} - f_{N-1} }{v_{N}-v_{N-1}}  \right]  \\
\end{equation}
\begin{equation}
   a_{N-1} = \frac{1}{J_N \left( x_{N+1/2} - x_{N-1/2} \right)} \left[- \frac{1}{2}\left(JA\right)_{N-1/2} + \frac{\left(JD\right)_{N-1/2} }{ x_{N}-x_{N-1}} \right]
\end{equation}
\begin{equation}
   b_{N} = \frac{1}{J_N \left( x_{N+1/2} - x_{N-1/2} \right)} \left[ - \frac{1}{2}\left(JA\right)_{N-1/2} - \frac{\left(JD\right)_{N-1/2} }{ x_{N}-x_{N-1}} \right]
\end{equation}
\begin{equation}
   \Delta S_N  = \frac{- J_{N+1/2} \Gamma_{N+1/2}}{J_N \left( x_{N+1/2} - x_{N-1/2} \right)}.
\end{equation}







\end{document}
